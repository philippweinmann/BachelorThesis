\chapter{Conclusions and Outlook}
\label{ch:outlook}
A lot of work is still ahead of us, but scientists are starting to tackle artificial intelligence with a more open approach, providing fast progress in the field. It is essential to stop trying to analyse and improve machine learning algorithms from a purely mathematical standpoint if we somehow want to stay in control of our systems or want to be able to debug them in the future.

\section{Interactivity}

 Human psychology is a very tricky subject to approach, because it can be extremely subjective. While some properties are common to everyone, others are dependent on your culture or experience. I believe that it is imperative, that if we use psychological findings to set parameters in our tools, we need to allow the person using those tools to adapt those parameters to fit not the average human but himself. Like scissors don't come in the average human hand size, but in multiple ones, so should our explainable artificial intelligence not provide one explanation for all of humanity but multiple ones. There have been attempts to implement interactivity in XAI, but they are in their infancy. Based on our findings, the text hierarchy used to provide an explanation for text can be a very useful tool and should be a parameter the XAI user can tweak.

\section{The future of XAI}

XAI will keep becoming more and more of an essential tool. The stronger our computational capacities become, the more complex algorithms can be run. These will impact our lives to unimaginable extent and can only be developed and controlled by developing XAI alongside. While that might seem challenging, I am confident that if we start taking into account interdisciplinary ideas like psychology and philosophy, XAI can make important leaps forward to catch up to where it needs to be.
